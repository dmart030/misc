\documentclass{article}
\usepackage{amsmath,amssymb,graphicx,setspace}
\begin{document}
Dominic Martinez-Ta\\
Physics 530A
\section{}
Two extremely long, cylindrical conductors of radii $a_1$ and $a_2$ are paralle and separated by a distance d which is very large compare to $a_1$ and $a_2$, but small compared to the length (so you can ignore end effects).
\begin{enumerate}
\item Show that the two-conductor capacitance per capacitance per unit length is given approximately by 
\begin{equation}
C \simeq \pi \epsilon_0 (ln\frac{d}{a})^{-1}
\end{equation}
where $a = \sqrt{a_1a_2}$.
Note: the condition  $d \gg a_1, a2$ allows you to use superposition to determine the total field/ potential produced by the combination of the two capacitors (they are far enough apart to have negligable effectsr on each other's charge distributions). 
Note 2: If we put $+Q$ and $-Q$ on the two conductors and the resulting potential difference between the two is $\Delta V$, then the capacitance is $C=\frac{Q}{\Delta V}$
\end{enumerate}
\section{solutions}
\begin{equation}
E_{tot} = E_{a_1} + E_{a_2} = \frac{Q}{2\pi\epsilon_0 L} + \frac{Q}{2\pi\epsilon_0 (d-L)} = \frac{Q}{2\pi\epsilon_0}(\frac{1}{L}+\frac{1}{d-L})
\end{equation}
\begin{equation}
V = \int_{a_1}^{d-a_2}E dl
\end{equation}
This basically becomes (via mathematica):
\begin{equation}
V = \frac{Q}{2\pi\epsilon_0}ln(\frac{(d-a_1)(d-a_2)}{a_1a_2}) \Rightarrow \frac{Q}{2\pi\epsilon_0}2ln(\frac{d}{a_1a_2})\Rightarrow \frac{Q}{\pi\epsilon_0}2n(\frac{d}{a_1a_2})
\end{equation}
Since $a = a_1a_2$, we can say that
\begin{equation}
 \frac{Q}{\pi\epsilon_0}ln(\frac{d}{a_1a_2})\Rightarrow \frac{Q}{\pi\epsilon_0}ln(\frac{d}{a})
\end{equation}
Since we know by definition that $C = \frac{Q}{V}$, this equation then becomes:
\begin{equation}
C \simeq \frac{2\pi\epsilon_0}{ln(\frac{d}{a})}
\end{equation}
QED
\end{document}